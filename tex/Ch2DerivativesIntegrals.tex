\chapter{Derivatives, Integrals, and Geometric Calculus}

	\section{The Geometric Derivative}
	
	Geometric Algebra is extended to Geometric Calculus by adding geometric derivatives and integrals. The geometric derivative is denoted by $\nabla$.	
	\[
	\nabla F\left( x\right)  = e_i\partial_i F \left( x\right).
	\]	
	The following sections demonstrate the geometric derivative in various spaces to give a practical sense of its usage. Then, we examine discrete \emph{simplicial} calculus and show how its limits produce the geometric derivative and fundamental calculus theorems like Stokes' and Cauchy's Theorems. Finally, we examine the theory of Green's functions. 
	
	\subsection{Euclidean 2-Space}
	
	\subsubsection{Real Functions}
	
	Consider a real-valued function $f$ over $x$ and $y$ in Euclidean 2-space. Its geometric derivative is	
	\[
	\nabla f = \left( \partial_x f\right) e_1  + \left( \partial_y f\right) e_2,
	\]	
	i.e., the geometric derivative of a real-valued function is its gradient.
	
	\subsubsection{Vector Functions}
	
	Consider a vector-valued function $g = u e_1 + ve_2.$ It's derivative is
	\begin{align*}
	\nabla g & = \left(e_1\partial_x + e_2\partial_y \right) \left(ue_1 + ve_2 \right) \\
	 & = \left( \partial_x u + \partial_y v\right) + \left( \partial_x v - \partial_y u\right)e_1e_2. 
	\end{align*}	
	In other words, the derivative of a vector-valued function in Euclidean 2-space is the complex derivative. In particular, we see that the geometric derivatives separates into inner (gradient) and outer (curl) products,
	\[
	\nabla g = \nabla \cdot g + \nabla \wedge g.
	\]
	The geometric derivative contains both divergence and curl from traditional vector calculus.
	
	\subsubsection{Bivector (Pseudoscalar) Functions}
	
	For $fe_1e_2$,
	\[
	\nabla f = \left(\partial_x f\right) e_2 - \left(\partial_y f\right) e_1.
	\]	
	\subsection{Euclidean 3-Space}
	
	\subsubsection{Real Functions}
	
	The geometric derivative acting on a real-valued function $f$ in 3-space is
	\[
	\nabla f = \left( \partial_x f\right) e_1  + \left( \partial_y f\right) e_2 + \left( \partial_z f\right) e_3.
	\]
	Again, this is simply the gradient of $f$.
	
	\subsubsection{Vector Functions}
	
	Given a vector-valued function $g = u e_1 + ve_2 + we_3.$ Its derivative is	
	\begin{align*}
	\nabla g & = \left(e_1\partial_x + e_2\partial_y + e_3\partial_z\right) \left(ue_1 + ve_2 + we_3\right) \\
	& = \left( \partial_x u + \partial_y v + \partial_z w\right) + \\
	& = \left( \partial_x v - \partial_y u\right)e_1e_2 + \\
	& = \left( \partial_y w - \partial_z v\right)e_2e_3 + \\
	& = \left( \partial_z u - \partial_x w\right)e_3e_1.
	\end{align*}	
	\subsubsection{Bivector Functions}
	A bivector function in 3D Euclidean space has the form
	\[
	f = ue_1e_2 + ve_2e_3 + we_3e_1.
	\]	
	Its derivative is	
	\begin{align*}
	\nabla f &= \left(e_1\partial_x + e_2\partial_y + e_3\partial_z\right) \left(ue_1e_2 + ve_2e_3 + we_3e_1\right) \\
	&= \left(\partial_zw - \partial_yu\right)e_1 \\
	&+ \left(\partial_xu - \partial_zv\right)e_2\\
	&+ \left(\partial_yv - \partial_xw\right)e_3\\
	&+ \left(\partial_xv + \partial_yw + \partial_zu\right)e_1e_2e_3\\
	\end{align*}
	\subsubsection{Pseudoscalar Functions}
	
	For $fe_1e_2e_3,$	
	\[
	\nabla f = \left(e_1\partial_x + e_2\partial_y + e_3\partial_z\right) f = \partial_x fe_2e_3 - \partial_yfe_1e_2 + \partial_zf e_1e_2. 
	\]	
	\subsection{Minkowski Spacetime}
	\subsubsection{Real Functions}
	\subsubsection{Vector Functions}
	\subsubsection{Bivector Functions}
	\subsubsection{Pseudovector Functions}
	\subsubsection{Pseudoscalar Functions}
	\subsection{Conformal 3-Space}
	
	\section{Gauge Covariant Derivative}
	
	In field theories, symmetries of the Lagrangian for the system determine the kinematic equations for that system. Consider a field theory in $\psi\left(x\right)$ with the following Lagrangian,	
	\[
	\mathcal{L} = \overline{\psi} D \psi.
	\]	
	Here, $D$ is a derivative operator. We'd like this Lagrangian to be invariant under the following field transformations,	
	\[
	\psi' = e^{-\lambda}\psi,
	\]	
	so that	
	\[
	\mathcal{L} = \overline{\psi} D \psi = \overline{\psi}' D' \psi'.
	\]	
	Transformations can be \emph{global}, where $\lambda$ is constant, or \emph{local}, where $\lambda$ is a function of $x$. We consider local transformations here. Expanding the right side,	
	\[
	\overline{\psi}' D' \psi' = \overline{\psi}e^{\lambda} D'e^{-\lambda}\psi.
	\]	
	So $D$ transforms $D' = e^{-\lambda} De^{\lambda}.$
	
	If we let $D = \partial$, the derivative product rule breaks invariance,	
	\begin{align*}
	\mathcal{L} &= \overline{\psi}' D \psi' \\
	&= \overline{\psi}' \partial \psi' \\
	&= \overline{\psi}e^{\lambda} \partial e^{-\lambda}\psi \\
	&= \left(\overline{\psi}e^{\lambda}\right) e^{-\lambda}\left(\left(\partial\psi\right) - \left(\partial\lambda\right)\psi\right) \\
	&= \overline{\psi}\partial\psi - \overline{\psi}\left(\partial\lambda\right)\psi.
	\end{align*} 	
	There's an extra $\overline{\psi}\left(\partial\lambda\right)\psi$ term, the \emph{gauge} term. The usual way to offset this is to add a \emph{gauge field} $A$ to $D$, then determine its transformation properties. Letting $D = \partial + A$, invariance requires	
	\begin{align*}
	D' &= \partial + A' \\
	&= e^{-\lambda}De^{\lambda}\\
	&=e^{-\lambda}\left(\partial + A\right)e^{\lambda} \\
	&= e^{-\lambda}\left(e^{\lambda}\partial\lambda + e^{\lambda}\partial + Ae^{\lambda}\right)\\
	&= \partial + \left(e^{-\lambda}Ae^{\lambda}+\partial\lambda\right).
	\end{align*}	
	This shows the gauge field transforms $A' = e^{-\lambda}Ae^{\lambda}+\partial\lambda.$
	
	\section{Simplices}
	
	To derive integral formulas for geometric calculus, we need to cast the regions over which functions are integrated into the formalism of geometric algebra. For example, the 1D integral of $f$ along a path $\gamma$ is written
	\[
	\int_\gamma f\;dx.
	\]
	It's important to note that the differential $dx$ encodes a \emph{direction} along the positive $x$ axis. Reversing the direction of $dx$ reverses the sign of entire integral.
	
	Similarly, if a function $g$ is defined over two-dimensional region $\Omega$, its integral is
	\[
	\int_\Omega g\;dxdy.
	\]
	Again, the differential $dxdy$ has a direction. Reversing it, $dydx = -dxdy$, changes the overall sign of the integral. This suggests that regions over which functions are integrated carry geometric content.
	
	A \emph{simplex} $a\left(k\right)$ is an oriented $k$-dimensional volume defined by the set of $k + 1$ points $\left(a_0, ... , a_k\right)$. In one dimension, a simplex $a\left(1\right)$ is a directed line segment between two points, $a_1 - a_0$, with volume $|a_1 - a_0|$.
	
	In two dimensions, 	
	\[
	a\left(2\right) = \left(a_1 - a_0\right)\wedge\left(a_2 - a_0\right)
	\]
	Its volume, 
	\[
	|a\left(2\right)| = \frac{1}{2}|\left(a_1 - a_0\right) \wedge \left(a_2 - a_0\right)|.
	\]
	Generally,
	\[
	a\left(k\right) = \left(a_1 - a_0\right)\wedge ... \wedge\left(a_k - a_0\right)
	\]
	and volume,
	\[
	|a\left(k\right)| = \frac{1}{k!}|\left(a_1 - a_0\right) \wedge...\wedge \left(a_k - a_0\right)|.
	\]
	For tidiness of notation, let $\overline{a_k} = a_k - a_0$, or an edge of a simplex. A simplex in two dimensions can be written
	\[
	a\left(2\right) = \frac{1}{2}\;\overline{a_1} \wedge \overline{a_2}. 
	\] 	
	A \emph{chain} of simplices $\mathcal{C}$ is a collection simplices $a^j\left(k\right)$, typically with edges or faces shared amongst simplices to form connected regions.
	
	We can approximate an integral of a function $g$ over a region $\Omega$ by breaking $\Omega$ up into a chain of simplices and taking the product of $g$ with each simplex $a^j$, evaluated at one point on that simplex,
	\[
	\sum g\left(a^j_0\right) a^j.
	\]
	Consider the following sum of $f=u\;e_1 + v\;e_2$ over a square region $\Omega$ with boundary $\partial\Omega$ formed the by the chain $\mathcal{C}$ formed by simplices
	\begin{align*}
	a^0 &= \Delta x\;e_1 = \overline{a_1}^0 \\
	a^1 &= \Delta y\;e_2 =\overline{a_1}^1 \\
	a^2 &= -\Delta x\;e_1 =\overline{a_1}^2 \\
	a^3 &= -\Delta y\;e_2 =\overline{a_1}^3
	\end{align*}
	Summing $f$ over this region,
	\[
	\sum_\mathcal{C} f = \sum f\left(a^i_0\right) \overline{a_1}^i.	
	\]
	Expanding the sum,
	\begin{align*}
	\sum_\mathcal{C} f &= \left(u\;e_1 + v\;e_2\right)\Delta x\;e_1 \\
	&+ \left(u\;e_1 + v\;e_2\right)\Delta y\;e_2 \\
	&- \left(u\;e_1 + v\;e_2\right)\Delta x\;e_1 \\
	&- \left(u\;e_1 + v\;e_2\right)\Delta y\;e_2.
	\end{align*}
	
	\section{Cauchy's Integral Theorem and Formula}
	
	