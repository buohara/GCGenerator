\chapter{Appendix 1: Calculations and Derivations}

\section{STA Lagrangian Invariance under $U\left(1\right)$}

\section{Conserved Currents}

An important concept in physics is \emph{conservation}, particularly as stated by \emph{Noether's Theorem}. The theorem gives a relationship between symmetries in a system's Lagrangian and conserved quantities like momentum in rigid body mechanics, or probability and spin in quantum mechanics.

Consider a field theory involving a field $\psi$. The Lagrangian for the system is $\mathcal{L}\left(\psi,\partial\psi,x\right).$ First, use the variational principle to determine the equations of motion for this system. The action $S$ for this system is
\[
S = \int \mathcal{L}\left(\psi,\partial\psi,x\right).
\]
The equations of motion can be derived by setting the variation to zero,
\[
\delta S = \delta \int \mathcal{L}\left(\psi,\partial\psi,x\right) = 0.
\]

The first-order variation $\delta\mathcal{L}$ is
\begin{align*}
\delta\mathcal{L} &= \mathcal{L}\left(\psi + \delta\psi,\partial\psi + \delta \left(\partial\psi\right),x + \delta x\right) - \mathcal{L}\left(\psi,\partial\psi,x\right)\\
&= \frac{\partial\mathcal{L}}{\partial\psi}\delta\psi + \frac{\partial\mathcal{L}}{\partial\left(\partial\psi\right)}\delta \left(\partial\psi\right) + \frac{\partial\mathcal{L}}{\partial x}\delta x.
\end{align*}
For this discussion, assume $\partial\mathcal{L}/\partial x = 0$. The variation becomes
\begin{align*}
\delta S &= \int \left(\frac{\partial\mathcal{L}}{\partial\psi}\delta\psi + \frac{\partial\mathcal{L}}{\partial\left(\partial\psi\right)}\delta \left(\partial\psi\right)\right)\\
&=\int \left(\frac{\partial\mathcal{L}}{\partial\psi}\delta\psi + \frac{\partial\mathcal{L}}{\partial\left(\partial\psi\right)}\partial\left(\delta\psi\right)\right).
\end{align*}
In the second line, we've used equality of mixed partials, $\delta\left(\partial\psi\right) = \partial\left(\delta\psi\right).$

The two integrand terms are variations in $\delta\psi$ and $\delta\left(\partial\psi\right)$. They can be combined as terms in $\partial\psi$ using the product rule,
\[
\partial\left(\frac{\partial \mathcal{L}}{\partial\left(\partial\psi\right)}\delta\psi\right) = \partial\left(\frac{\partial \mathcal{L}}{\partial\left(\partial\psi\right)}\right)\delta\psi +
\frac{\partial \mathcal{L}}{\partial\left(\partial\psi\right)}\partial\left(\delta\psi\right),
\]
so
\[
\frac{\partial \mathcal{L}}{\partial\left(\partial\psi\right)}\partial\left(\delta\psi\right) = \partial\left(\frac{\partial \mathcal{L}}{\partial\left(\partial\psi\right)}\delta\psi\right) -
\partial\left(\frac{\partial \mathcal{L}}{\partial\left(\partial\psi\right)}\right)\delta\psi. 
\].
Substituting,
\[
\delta S = \int \left(\frac{\partial\mathcal{L}}{\partial\psi}\delta\psi + \partial\left(\frac{\partial \mathcal{L}}{\partial\left(\partial\psi\right)}\delta\psi\right) -
\partial\left(\frac{\partial \mathcal{L}}{\partial\left(\partial\psi\right)}\right)\delta\psi\right). 
\]
The middle term is vanishes since variations $\delta\psi$ are assumed to be zero on the integration region boundary,
\[
\int \partial\left(\frac{\partial \mathcal{L}}{\partial\left(\partial\psi\right)}\delta\psi\right) = \frac{\partial \mathcal{L}}{\partial\left(\partial\psi\right)}\delta\psi \bigg\rvert_{\partial\Omega} = 0.
\]
Finally, we're left with
\[
\delta S = \int \left(\frac{\partial\mathcal{L}}{\partial\psi} -
\partial\left(\frac{\partial \mathcal{L}}{\partial\left(\partial\psi\right)}\right)\right)\delta\psi.
\]
A zero variation gives the equations of motion,
\[
\frac{\partial\mathcal{L}}{\partial\psi} =
\partial\left(\frac{\partial \mathcal{L}}{\partial\left(\partial\psi\right)}\right).
\]
This equation of motion implies a relationship between symmetries and conserved quantities. If the Lagrangian is invariant under some field symmetry, i.e.,
\[
\frac{\partial\mathcal{L}}{\partial\psi} = 0,
\]
Then the following quantity is conserved,
\[
\partial\left(\frac{\partial \mathcal{L}}{\partial\left(\partial\psi\right)}\right) = 0.
\]