\chapter{Appendix 1: Calculations and Derivations}

\section{STA Lagrangian Invariance under $U\left(1\right)$}

\section{Conserved Currents}

An important concept in physics is \emph{conservation}, particularly as stated by \emph{Noether's Theorem}. The theorem gives a relationship between symmetries in a system's Lagrangian and conserved quantities like momentum in rigid body mechanics, or probability and spin in quantum mechanics.

Consider a field theory involving a field $\psi$. The Lagrangian for the system is $\mathcal{L}\left(\psi,\partial\psi,x\right).$ First, use the variational principle to determine the equations of motion for this system. The action $S$ for this system is
\[
S = \int \mathcal{L}\left(\psi,\partial\psi,x\right).
\]
The equations of motion can be derived by setting the variation to zero,
\[
\delta S = \delta \int \mathcal{L}\left(\psi,\partial\psi,x\right) = 0.
\]

The first-order variation $\delta\mathcal{L}$ is
\begin{align*}
\delta\mathcal{L} &= \mathcal{L}\left(\psi + \delta\psi,\partial\psi + \delta \left(\partial\psi\right),x + \delta x\right) - \mathcal{L}\left(\psi,\partial\psi,x\right)\\
&= \frac{\partial\mathcal{L}}{\partial\psi}\delta\psi + \frac{\partial\mathcal{L}}{\partial\left(\partial\psi\right)}\delta \left(\partial\psi\right) + \frac{\partial\mathcal{L}}{\partial x}\delta x.
\end{align*}
For this discussion, assume $\partial\mathcal{L}/\partial x = 0$. The variation becomes
\begin{align*}
\delta S &= \int \left(\frac{\partial\mathcal{L}}{\partial\psi}\delta\psi + \frac{\partial\mathcal{L}}{\partial\left(\partial\psi\right)}\delta \left(\partial\psi\right)\right)\\
&=\int \left(\frac{\partial\mathcal{L}}{\partial\psi}\delta\psi + \frac{\partial\mathcal{L}}{\partial\left(\partial\psi\right)}\partial\left(\delta\psi\right)\right).
\end{align*}
In the second line, we've used equality of mixed partials, $\delta\left(\partial\psi\right) = \partial\left(\delta\psi\right).$

The two integrand terms are variations in $\delta\psi$ and $\delta\left(\partial\psi\right)$. They can be combined as terms in $\partial\psi$ using the product rule,
\[
\partial\left(\frac{\partial \mathcal{L}}{\partial\left(\partial\psi\right)}\delta\psi\right) = \partial\left(\frac{\partial \mathcal{L}}{\partial\left(\partial\psi\right)}\right)\delta\psi +
\frac{\partial \mathcal{L}}{\partial\left(\partial\psi\right)}\partial\left(\delta\psi\right),
\]
so
\[
\frac{\partial \mathcal{L}}{\partial\left(\partial\psi\right)}\partial\left(\delta\psi\right) = \partial\left(\frac{\partial \mathcal{L}}{\partial\left(\partial\psi\right)}\delta\psi\right) -
\partial\left(\frac{\partial \mathcal{L}}{\partial\left(\partial\psi\right)}\right)\delta\psi. 
\].
Substituting,
\[
\delta S = \int \left(\frac{\partial\mathcal{L}}{\partial\psi}\delta\psi + \partial\left(\frac{\partial \mathcal{L}}{\partial\left(\partial\psi\right)}\delta\psi\right) -
\partial\left(\frac{\partial \mathcal{L}}{\partial\left(\partial\psi\right)}\right)\delta\psi\right). 
\]
The middle term is vanishes since variations $\delta\psi$ are assumed to be zero on the integration region boundary,
\[
\int \partial\left(\frac{\partial \mathcal{L}}{\partial\left(\partial\psi\right)}\delta\psi\right) = \frac{\partial \mathcal{L}}{\partial\left(\partial\psi\right)}\delta\psi \bigg\rvert_{\partial\Omega} = 0.
\]
Finally, we're left with
\[
\delta S = \int \left(\frac{\partial\mathcal{L}}{\partial\psi} -
\partial\left(\frac{\partial \mathcal{L}}{\partial\left(\partial\psi\right)}\right)\right)\delta\psi.
\]
A zero variation gives the equations of motion,
\[
\frac{\partial\mathcal{L}}{\partial\psi} =
\partial\left(\frac{\partial \mathcal{L}}{\partial\left(\partial\psi\right)}\right).
\]
These are called the \emph{Euler-Lagrange} equations. This equation of motion implies a relationship between symmetries and conserved quantities. If the Lagrangian is invariant under some field symmetry, i.e.,
\[
\frac{\partial\mathcal{L}}{\partial\psi} = 0,
\]
Then the following quantity is conserved,
\[
\partial\left(\frac{\partial \mathcal{L}}{\partial\left(\partial\psi\right)}\right) = 0.
\]

\section{Classical Conservation of Momentum}

The Lagrangian for a particle of mass $m$ in classical mechanics is
\[
\mathcal{L} = \frac{p^2}{2m} + V\left(x\right) = \frac{mv^2}{2}+ V\left(x\right).
\]
Applying the Euler-Lagrange equations,
\[
\frac{d\mathcal{L}}{d x} - \frac{d}{dt}\left(\frac{d\mathcal{L}}{d v}\right) = \frac{dV}{dx}- \frac{d}{dt}\left(mv\right) = \frac{dV}{dx} - \frac{d\left(mv\right)}{dt}.
\]
Requiring this to be zero,
\[
\frac{d\left(mv\right)}{dt} = \frac{dp}{dt} = \frac{dV}{dx},
\]
which is Newton's second law.

For a free particle, $V=0$, and momentum is conserved,
\[
\frac{d}{dt}\left(\frac{d\mathcal{L}}{dv}\right) = \frac{dp}{dt} = 0.
\]

\section{Relativistic Conservation of Energy-Momentum}
A relativistic spacetime interval is
\[
ds^2 = \left(dx^\mu\right)^2.
\]
The action for a free particle is
\[
S = \int \left(\left(dx^\mu\right)^2\right)^{1/2}
\]
with Lagrangian
\[
\mathcal{L} = \left(\left(dx^\mu\right)^2\right)^{1/2}.
\].
The Euler-Lagrange equations for this system are
\[
\frac{\partial\mathcal{L}}{\partial x^\mu} - \frac{d}{d\tau}\left( \frac{\partial\mathcal{L}}{\partial \left(dx^\mu\right)}\right).
\]
The first term is zero, and evaluating the second term,
\begin{align*}
\frac{d}{d\tau}\left( \frac{\partial\mathcal{L}}{\partial \left(dx^\mu\right)}\right) &= \frac{d}{d\tau} \left(\frac{dx^\mu}{\left(\left(dx^\mu\right)^2\right)^{1/2}}\right)\\ 
&= \frac{d}{d\tau}\left(\gamma dx^\mu\right) \\
&= \frac{dp}{d\tau} \\
&= 0.
\end{align*}
\section{Conservation of Probability in Schrodinger Theory}
The Schrodinger equation for a free particle is
\[
\frac{\partial\psi}{\partial t} = i\left(\frac{1}{2m}\nabla^2\right)\psi.
\]
Complex conjugation gives the adjoint Schrodinger equation,
\[
\frac{\partial\psi^*}{\partial t} = -i\left(\frac{1}{2m}\nabla^2\right)\psi^*.
\]
Probability normalization requires
\[
\int_\Omega \psi^*\psi \: dx = 1,
\]
which is conserved through time,
\[
\frac{\partial}{\partial t} \int_\Omega \psi^*\psi \: dx = \int_\Omega \left(\frac{\partial \psi^*}{\partial t}\psi + \psi^*\frac{\partial \psi}{\partial t}\right) \: dx = 0.
\]
Using the Schrodinger equation and its adjoint to substitute for $\partial \psi /\partial t$ and $\partial \psi^* /\partial t$,
\begin{align*}
\int_\Omega \left(\frac{\partial \psi^*}{\partial t}\psi + \psi^*\frac{\partial \psi}{\partial t}\right) \: dx & \\
= \frac{i}{2m}\int_\Omega\left(\psi^*\nabla^2\psi - \psi\nabla^2\psi^*\right) \: dx.
\end{align*}
This can be simplified noting
\begin{align*}
\nabla\left(\psi^*\nabla\psi - \psi\nabla\psi^*\right) & \\
= \left(\nabla\psi^*\right)\left(\nabla\psi\right) + \psi^*\left(\nabla^2\psi\right) - \left(\nabla\psi\right)\left(\nabla\psi^*\right) - \psi\left(\nabla^2\psi^*\right) &\\
=\psi^*\nabla^2\psi - \psi\nabla^2\psi^*.
\end{align*}
Substituting,
\begin{align*}
&\frac{i}{2m}\int_\Omega\left(\psi^*\nabla^2\psi - \psi\nabla^2\psi^*\right) \: dx \\
=&\: \frac{i}{2m}\int_\Omega\nabla\left(\psi^*\nabla\psi - \psi\nabla\psi^*\right) \: dx \\
=&\: \frac{i}{2m}\int_{\partial\Omega}\left(\psi^*\nabla\psi - \psi\nabla\psi^*\right) \: dx.
\end{align*}
\section{Derivative of General STA Versor}
An STA versor $\psi$ has 8 components,
\[
\psi = \alpha + \beta + \gamma,
\]
where $\alpha$ is a scalar, $\beta$ is a bivector, and $\gamma$ is a pseudoscalar.
