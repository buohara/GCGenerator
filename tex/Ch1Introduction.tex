\chapter{Introduction}
	
	The objects of a geometric algebra are called \emph{multivectors}. Multivectors generalize objects like directed lines, planes, and volumes. An important property of multivectors is they have \emph{orientation}, i.e., a sense of up/down, inside/outside, etc. The following sections introduce examples of geometric algebras and the operations on their multivectors.
	\section{2D Euclidean Space and $U\left( 1\right) $}
	\subsection{Geometric Algebra of Euclidean 2-Space}
	Multivectors are composed of \emph{basis one-vectors}. In Euclidean 2-space, these are unit vectors along the positive $x$ and $y$ axes. They are denoted by $e_1$ and $e_2$, respectively (the use of $e$ here is shorthand for Euclidean. Different symbols may be used for bases in other algebras, as will be seen later).
	
	The most important operation between multivectors is the \emph{geometric product}. If $a$ and $b$ are multivectors, the geometric product of $a$ and $b$ is written $ab$. The geometric product is the sum of two simpler products, the \emph{dot} (inner) and \emph{wedge} (outer) products. In Euclidean 2-space, these are equivalent to the familiar dot and cross products of vector algebra. The dot product of $a$ and $b$ is written $a \cdot b$ and the wedge product is written $a \wedge b$. The geometric product, then, is written
	\[
	ab = a \cdot b + a \wedge b.
	\]
	
	The inner products of basis one-vectors amongst themselves defines the \emph{signature} of an algebra,	
	\begin{align*}
	e_1 \cdot e_1 & = 1\\
	e_2 \cdot e_2 & = 1\\	
	e_1 \cdot e_2 & = 0
	\end{align*}	
	Basis one-vectors represent directed unit lines. An oriented plane can be made by wedging the one-vectors together to form a basis \emph{bivector}.	
	\begin{align*}		
	e_1 \wedge e_2 & = x\text{-}y\text{ plane, counterclockwise orientation }\\
	e_i \wedge e_j & = 0 \quad \left(i = j\right).
	\end{align*}
	The plane can be flipped by reversing the wedge product.
	\begin{align*}		
	e_2 \wedge e_1 & = -e_1 \wedge e_2 = x\text{-}y\text{ plane, clockwise orientation }
	\end{align*}
	More generally, dot products are \emph{symmetric} and wedge products are \emph{antisymmetric}
	\[
	\frac{1}{2}\left( ab + ba\right)  = \frac{1}{2}\left( a \cdot b + a \wedge b + b \cdot a + b \wedge a\right)  = a \cdot b,
	\]
	\[
	\frac{1}{2}\left( ab - ba\right)  = \frac{1}{2}\left( a \cdot b + a \wedge b - b \cdot a - b \wedge a\right)  = a \wedge b.
	\]
	
	Since unit vectors are \emph{orthogonal},
	\begin{align*}
	e_ie_j & = e_i \cdot e_j + e_i \wedge e_j = 1 & \left( i = j\right) \\
	e_ie_j & = e_i \cdot e_j + e_i \wedge e_j = e_i \wedge e_j & \left( i \ne j\right) 
	\end{align*}
	explicit dots and wedges are unnecessary when writing basis vectors. We simply write $e_1e_2$ instead of $e_1 \wedge e_2$.
	
	A basis \emph{zero-vector} is a scalar.
	
	A general multivector $m$ in Euclicean 2-space is a linear combination of basis vectors,
	\[
	m = s + a_1e_1 + a_2e_2 + be_1e_2.
	\]
	The following formula is seldom used\footnote{One application is coding computer algebra systems.}, but for completeness, the product of two general multivectors,
	\begin{align*}
	m &= s + a_1e_1 + a_2e_2 + be_1e_2,\\
	n &= r + c_1e_1 + c_2e_2 + de_1e_2
	\end{align*}
	is
	\begin{align*}
	mn &= \left( rs + a_1b_1 + a_2b_2 - bd\right) \\
	&+ \left( ra_1 + sc_1 - a_2d + c_2b\right) e_1\\
	&+ \left( ra_2 + sc_2 + a_1d - c_1b\right) e_2\\
	&+ \left( a_1c_2 - a_2c_1\right) e_1e_2.
	\end{align*}
	
	The basis vector formed by multiplying all basis one-vectors is called the \emph{unit pseudoscalar} and is denoted by $I$. In 2-space, $I = e_1e_2$. This is in direct analogy to $i = \sqrt{-1}$ from complex numbers, as shown below.
	
	Often, we'll be interested in \emph{even} multivectors, i.e., linear combinations of zero-vectors, bivectors, four-vectors, etc. The product of even multivectors
	\begin{align*}
	m &= a + be_1e_2 = a + bI,\\
	n &= c + de_1e_2 = c + dI
	\end{align*}
	is
	\[
	mn = \left( ac - bd\right)  + \left( ad + bc\right) I,
	\]
	which is the formula for multiplying two complex numbers. From this perspective, the $i$ from complex algebra can be thought of as a counterclockwise-oriented plane.
	
	Unit psuedoscalars satisfy $I^2 = -1$. For this, we define the \emph{reverse} operator on multivectors, which reverses the order of basis vectors. The reverse of $I = e_1e_2$ in 2-space is
	\[
	\tilde{I} = e_2e_1 = -I.
	\]
	Note that an odd number of swaps (just 1 in euclidean 2-space) is required to reverse $I$, so the signature of the algebra doesn't require modification to satisfy $I^2 = -1.$ In other algebras, like Minkowski spacetime, the signature will need modification to satisfy this requirement.
	
	The square of a multivector is defined by multiplying a multivector by its reverse. For example,
	\[
	I^2 = \tilde{I}I = e_3e_2e_1\left( e_1e_2e_3\right)  = -1.
	\]
	We can use $I$ to compute the \emph{dual} of a multivector simply by multiplying. The dual $M$ of a multivector $m$ is
	\[
	M = Im.
	\]
	For example,
	\[
	Ie_2 = \left( e_1e_2\right) e_2 = e_1.
	\]
	If $m$ spans a subspace of Euclidean 2-space, its dual spans the remaining subspace needed to fill out 2-space. This is the same as the orthogonal complement in linear algebra.
	
	\subsection{$U\left( 1\right) $ as a Geometric Algebra}
	In later chapters, we'll discuss symmetries in field theories. Many of these symmetries involve \emph{unitary} groups. We show here and in later sections how unitary groups can be represented with geometric algebras. 
	
	Unitary groups $U\left( n\right) $ are groups of $n \times n$ unitary matrices, i.e., matrices $U$ where $U^\dagger U = I$. 
	
	For $n = 1$, this is simply the group of unit complex numbers. As shown above, this group equivalent to the group of even multivectors in Euclidean 2-space of unit magnitude.
	
	Unitary groups have \emph{Lie algebras} and \emph{generators}.
	
	\section{3D Euclidean Space and $SU\left( 2\right) $}
	\subsection{Geometric Algebra of Euclidean 3-Space}
	In Euclidean 3-space, the three basis vectors are $e_1$, $e_2$, and $e_3$. Their inner products satisfy,	
	\begin{align*}
		e_1 \cdot e_1 & = 1\\
		e_2 \cdot e_2 & = 1\\
		e_3 \cdot e_3 & = 1\\		
		e_i \cdot e_j & = 0 \quad \left(i\neq j\right)
	\end{align*}	
	The wedge products are,	
	\begin{align*}		
		e_1 \wedge e_2 & = x\text{-}y\text{ plane, normal along }+z\\
		e_2 \wedge e_3 & = y\text{-}z\text{ plane, normal along }+x\\
		e_3 \wedge e_1 & = z\text{-}x\text{ plane, normal along }+y\\
		e_i \wedge e_j & = 0 \quad \left(i=j\right)
	\end{align*}
	Flipping the planes,
	\begin{align*}		
	e_2 \wedge e_1 & = -e_1 \wedge e_2 = x\text{-}y\text{ plane, normal along }-z\\
	e_3 \wedge e_2 & = -e_2 \wedge e_3 = y\text{-}z\text{ plane, normal along }-x\\
	e_1 \wedge e_3 & = -e_3 \wedge e_1 = z\text{-}x\text{ plane, normal along }-y.
	\end{align*}
	
	The unit volume/pseudoscalar is,
	\[
	I = e_1e_2e_3 = e_1 \wedge e_2 \wedge e_3.  
	\]
	It's reverse is
	\[
	\tilde{I} = e_3e_2e_1 = -I.
	\]
	Again, an odd number of swaps (3) is required to reverse $I$, so no signature modification is required to satisfy $I^2=-1.$
	
	A general multivector $m$ in Euclicean 3-space is a scalar plus a linear combination of basis vectors,
	\[
	m = s + a_1e_1 + a_2e_2 + a_3e_3 + b_1e_2e_3 + b_2e_3e_1 + b_3e_1e_2 + ce_1e_2e_3.
	\] 
	\subsection{$SU\left( 2\right) $ as a Geometric Algebra}
	
	\section{Minkowski Spacetime}
	
	Minkowski spacetime has four dimensions, one timelike, three spacelike. These are denoted by $\gamma_i$ instead of $e_i$. The algebra's signature needs modification to satisfy $I^2 = -1$. Given $I = \gamma_0\gamma_1\gamma_2\gamma_3,$
	\[
	\tilde{I} = \gamma_3\gamma_2\gamma_1\gamma_0 = I.
	\]
	An even number of swaps (6) is required to reverse $I$. So, to satisfy $I^2=-1$, we need to modify the algebra's signature to achieve an overall negative sign. This can be done by setting the squares of 1 or 3 basis vectors to be negative. We choose the signature,
	\begin{align*}
	\gamma_0 \cdot \gamma_0 & = 1\\
	\gamma_1 \cdot \gamma_1 & = -1\\
	\gamma_2 \cdot \gamma_2 & = -1\\		
	\gamma_3 \cdot \gamma_3 & = -1\\
	\gamma_i \cdot \gamma_j & = 0 \quad \left(i\neq j\right).
	\end{align*}
	
	\section{Conformal 3-Space}
	
	Conformal algebras are created by taking an underlying space and adding a spacelike and timelike dimension to it. If we append $e_0$ (timelike) and $e_4$ (spacelike) to Euclidean 3-space, we have the basis $e_0$, $e_1$, $e_2$, $e_3$, $e_4$. To determine the signature, reverse $I$,
	  
	\[
	\tilde{I} = e_4e_3e_2e_1e_0 = I.
	\]
		
	This requires 10 swaps, so one, three, or five basis vectors should carry a negative sign. We choose $e_0$,
	\begin{align*}
	e_0 \cdot e_0 & = -1\\
	e_1 \cdot e_1 & = 1\\
	e_2 \cdot e_2 & = 1\\
	e_3 \cdot e_3 & = 1\\		
	e_4 \cdot e_4 & = 1\\
	e_i \cdot e_j & = 0 \quad \left(i\neq j\right)
	\end{align*}	
	The wedge products are,	
	\begin{align*}		
	e_1 \wedge e_2 & = x\text{-}y\text{ plane, normal along }+z\\
	e_2 \wedge e_3 & = y\text{-}z\text{ plane, normal along }+x\\
	e_3 \wedge e_1 & = z\text{-}x\text{ plane, normal along }+y\\
	e_i \wedge e_j & = 0 \quad \left(i=j\right)
	\end{align*}
	Flipping the planes,
	\begin{align*}		
	e_2 \wedge e_1 & = -e_1 \wedge e_2 = x\text{-}y\text{ plane, normal along }-z\\
	e_3 \wedge e_2 & = -e_2 \wedge e_3 = y\text{-}z\text{ plane, normal along }-x\\
	e_1 \wedge e_3 & = -e_3 \wedge e_1 = z\text{-}x\text{ plane, normal along }-y.
	\end{align*}	